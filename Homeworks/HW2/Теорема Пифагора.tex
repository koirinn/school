\documentclass{article}
\usepackage{graphicx} % Required for inserting images
\usepackage[russian]{babel}
\usepackage{hyperref}
\usepackage[width = 14cm, height = 22cm]{geometry}


\title{Мини конспект по теме: Теорема Пифагора}
\author{Маргарита Кирьянова}
\date{Сентябрь 2025г.}

\begin{document}

\maketitle
\tableofcontents

\newpage
\section{Введение}
\begin{flushleft}
    Теорема Пифагора — одна из важнейших теорем евклидовой геометрии. Она находит применение в самых разных областях:
\end{flushleft}
\begin{itemize}
    \item геометрия и тригонометрия
    \item физика
    \item инженерные расчёты
    \item компьютерная графика
\end{itemize}

\section{Формулировка теоремы}
\begin{flushleft}
    \textbf{Слова:} В прямоугольном треугольнике квадрат гипотенузы равен сумме квадратов катетов.
\end{flushleft}
\begin{equation}
    c^2 = a^2 + b^2
\end{equation}  
\begin{center}
    Как видно из формулы 1, знание двух сторон позволяет найти третью.
\end{center}

\section{Доказательство (набросок)}
\begin{quote}
    Одно из доказательств основывается на площади квадрата, составленного
 из четырёх одинаковых прямоугольных треугольников и малого квадрата
 в центре. Раскладывая площадь двумя способами, получаем $c^2 = a^2 +b^2$.
\end{quote}
% \end{flushleft}


\section{Примеры расчёта}
\subsection*{Пример 1}
\begin{center}
    $a = 3$, $b = 4$
\end{center}
\begin{center}   
    $c = \sqrt{a^2 + b^2} = \sqrt{9 + 16} = 5$
\end{center}
\subsection*{Пример 2}
\begin{quote}
1. Дано: $a = 5$, $b = 12$ \newline
\newline
2. Решениие: 
\end{quote}
\begin{center}
    $c = \sqrt{5^2 + 12^2} = \sqrt{25 + 144} = 13$
\end{center}

\section{Таблица значений}
\begin{center}
    \begin{tabular}{|c|c|c|}
    \hline
    Катет \textit{a} & Катет \textit{b} & Гипотенуза \textit{c} \\
    \hline
    3 & 4 & 5 \\
    \hline
    5 & 12 & 13 \\
    \hline
    7 & 24 & 25 \\
    \hline
    \end{tabular}
\end{center}

\newpage

\section{Иллюстрация}
\begin{flushleft}
    Ниже пример изображения
\end{flushleft}
\begin{center}
    \includegraphics[width=0.55\textwidth]{triangle.png}
\end{center}

\section{Заключение}
\begin{flushleft}
    Теорема Пифагора — один из краеугольных камней геометрии, помогающий решать множество практических задач.
\end{flushleft}

\section{Ссылки и литература}
\begin{itemize}
    \item Википедия: \href{https://ru.wikipedia.org/wiki/%D0%A2%D0%B5%D0%BE%D1%80%D0%B5%D0%BC%D0%B0_%D0%9F%D0%B8%D1%84%D0%B0%D0%B3%D0%BE%D1%80%D0%B0}{Теорема Пифагора}
    \item Классические учебники геометрии
\end{itemize}
\end{document}
